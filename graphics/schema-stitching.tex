\begin{tikzpicture}[
    node distance=0.5cm,
    component/.style = {
        draw,
        rectangle,
        minimum width = 3.5cm,
        align = center,
        minimum height = 1cm
    },
    process/.style = {
        draw,
        rectangle
    },
    functionality/.style = {
        draw,
        circle,
        minimum width = 5mm
    },
    server/.style = {
        draw,
        rectangle
    },
    sline/.style = {
        line cap=round,
        line join=round
    },
    database/.style = {
        draw,
        cylinder,
        shape border rotate=90,
        aspect=0.25,
        text width = 0.5cm,
        minimum height = 1cm,
        align = center,
        cylinder uses custom fill, cylinder body fill=white, cylinder end fill=black!5
    },
    table/.style = {
        draw,
        rectangle,
        minimum width=5mm,
        minimum height=5mm
    }
]

\node [component] (api1) {GraphQL API};
\node [component, right=of api1] (api2) {GraphQL API};
\node [component, right=of api2] (api3) {GraphQL API};

\node[component, below=0 of api1, minimum height=2cm, anchor=north, text depth=1cm] (bs1) {Data Source};
\node[component, below=0 of api2, minimum height=2cm, anchor=north, text depth=1cm] (bs2) {Data Source};
\node[component, below=0 of api3, minimum height=2cm, anchor=north, text depth=1cm] (bs3) {Data Source};

\node[database] at ($(bs1) - (0,4mm)$) {~};
\node[database] at ($(bs2) - (0,4mm)$) {~};
\node[database] at ($(bs3) - (0,4mm)$) {~};

\node[component, above=1cm of api2, minimum width=11.5cm] (stitching) {Gateway};
\node[component, above=0 of stitching, minimum width=11.5cm] (api) {GraphQL API};
\node[component, above=of api, minimum width=11.5cm] (client) {Client};

\draw[->] (api) -- (client);
\draw[->] (api2) -- (stitching);
\draw (api1) |- ($(api2)!.5!(stitching)$);
\draw (api3) |- ($(api2)!.5!(stitching)$);


\end{tikzpicture}