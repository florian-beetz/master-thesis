\section{Introduction}\label{sec:introduction}

% context
When you search for ``GraphQL vs. \acs{REST}'' on Google you get just shy of 4 million results with countless blog articles comparing the technology Facebook published in 2016 and the \acf{REST} architectural style~\cite{Brito2019}.
This shows that there is an immense interest in how these technologies compare.
Both technologies can be used to implement web service architectures.
GraphQL is designed to query web \acp{API}.
To illustrate how GraphQL can be used, suppose using the \ac{REST}-based \ac{SWAPI}\footnote{\url{https://swapi.dev/}} to obtain all titles of films where Luke Skywalker appeared in.
One would first send a request to the \textit{people} endpoint which returns a complex \ac{JSON} object containing 16 keys describing properties of Luke Skywalker and among them a list of the \acp{URL} to the film objects of the \ac{API}.
Then a request would be sent per film \ac{URL} which again returns a \ac{JSON} object with 14 keys, although only one is of interest to the consumer.
The GraphQL wrapper around the \ac{API}\footnote{\url{https://graphql.org/swapi-graphql/}} allows combining these in total six requests to one which returns exactly the data of interest:
\begin{lstlisting}[language=graphql]
query { person(personID: 1) { filmConnection { films { title } } } }
\end{lstlisting}


% problem including the research question

However, although there is an immense amount of gray literature on comparing \ac{REST} and GraphQL, almost no academic literature exists on this topic.
This thesis aims at giving guidelines for deciding which of these technologies is appropriate under which circumstances.
To achieve this goal, the architectural differences that arise when implementing a microservice architecture with \ac{REST} and GraphQL \acp{API} are evaluated.
The following research questions will be answered in this thesis.

\begin{description}
    \item[(\acs{RQ}1)] What are the architectural differences when implementing a microservice architecture with \ac{REST} and GraphQL \ac{API}?
    \item[(\acs{RQ}2)] How do microservices implemented with \ac{REST} and GraphQL \acp{API} compare performance-wise?
    \item[(\acs{RQ}3)] What are the differences when extending the schema of \acp{API} of microservices implemented with \ac{REST} and GraphQL interfaces?  
\end{description}

% why the problem is important and relevant

Providing such guidelines with an academic basis is especially important given the rising interest in GraphQL since it was published in 2016.
\autoref{fig:search-interest} shows, that according to Google Trends the interest in GraphQL constantly rose.
Although it is by far not as popular as \ac{REST}, this could also be caused by the recent release of this emerging technology.
However, this makes it even more important to not only analyze GraphQL in the immense amount of gray literature mentioned before but also to contribute to academic points of view.

\begin{figure}[h!]
    \centering
    \begin{tikzpicture}
        \begin{axis}[
            width=\linewidth, % Scale the plot to \linewidth
            height=5cm,
            grid=major, 
            grid style={dashed,gray!30},
            xlabel=Time, % Set the labels
            ylabel=Interest,
            legend columns=-1,
            legend style={/tikz/every even column/.append style={column sep=0.5cm}, at={(0.5,-0.6)},anchor=north},
            %legend pos=north west,
            x tick label style={rotate=90,anchor=east,font=\footnotesize},
            date coordinates in=x,
            date ZERO=2016-01-03,
            ymin=0,
            ymax=100,
            xmin=2016-01-03,
            xmax=2020-03-08
          ]
          \addplot[theme1] table[x=Woche,y=REST,col sep=comma] {data/search-interest.csv}; 
          \addlegendentry{REST}
          \addplot[theme2] table[x=Woche,y=GraphQL,col sep=comma] {data/search-interest.csv}; 
          \addlegendentry{GraphQL}
        \end{axis}
    \end{tikzpicture}
    \caption{Interest for \acs{REST} and GraphQL from Google Trends}\label{fig:search-interest}    
\end{figure}

% why the problem is non trivial
% why already available work is insufficient
Two technologies can be compared either by qualitative metrics or quantitative metrics.
This thesis contributes a comparison mainly based on qualitative metrics, however, it also builds on the previous work comparing quantitative metrics of both technologies~\cite{Brito2019,Brito2020,Wittern2018,Seabra2019}.
While qualitative comparisons were performed previously, they mostly focus on the implementation of clients for \acp{API}~\cite{Brito2020}.
Additionally, literature exists on migrating an existing \ac{REST} \ac{API} to GraphQL~\cite{Vogel2017, Lama2019, Wittern2018}.
However, the literature is lacking comparisons for scenarios of implementing \acp{API} in microservice architectures from scratch.

% how you are planning to solve this

This thesis first gives a theoretical overview of microservices and microservice architectures, \ac{REST}, and GraphQL in \autoref{sec:foundations}.
Then, related literature is presented in \autoref{sec:related}.
First, \autoref{sec:case-study} gives an overview of the setting of the case study performed as a part of this thesis, followed by the implementation details of its \ac{REST} version (see \autoref{sec:cs-rest}) and of its GraphQL version (see \autoref{sec:cs-graphql}).
Both versions are then compared by architectural differences in \autoref{sec:cs-comp} and are analyzed based on their performance in \autoref{sec:cs-perf}.
The required effort to introduce changes in both versions is analyzed in \autoref{sec:cs-schema-ev}.
This section ends with a discussion of the results in \autoref{sec:cs-disc}.
Lastly, the thesis ends with future work and the conclusion in \autoref{sec:conclusion}.
