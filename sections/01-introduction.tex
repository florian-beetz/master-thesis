\section{Introduction}\label{sec:introduction}

\begin{figure}[h!]
    \centering
    \begin{tikzpicture}
        \begin{axis}[
            width=\linewidth, % Scale the plot to \linewidth
            height=5cm,
            grid=major, 
            grid style={dashed,gray!30},
            xlabel=Time, % Set the labels
            ylabel=Interest,
            legend columns=-1,
            legend style={/tikz/every even column/.append style={column sep=0.5cm}, at={(0.5,-0.6)},anchor=north},
            %legend pos=north west,
            x tick label style={rotate=90,anchor=east,font=\footnotesize},
            date coordinates in=x,
            date ZERO=2016-01-03,
            ymin=0,
            ymax=100,
            xmin=2016-01-03,
            xmax=2020-03-08
          ]
          \addplot[theme1] table[x=Woche,y=REST,col sep=comma] {data/search-interest.csv}; 
          \addlegendentry{REST}
          \addplot[theme2] table[x=Woche,y=GraphQL,col sep=comma] {data/search-interest.csv}; 
          \addlegendentry{GraphQL}
        \end{axis}
    \end{tikzpicture}
    \caption{Interest for \acs{REST} and GraphQL from Google Trends}\label{fig:search-interest}    
\end{figure}

\begin{description}
    \item[(RQ1)] What are the architectural differences when implementing a microservice architecture with \ac{REST} and GraphQL \ac{API}?
    \item[(RQ2)] How do microservices implemented with \ac{REST} and GraphQL \acp{API} compare performance-wise?
    \item[(RQ3)] What are the differences when extending the schema of \acp{API} of microservices implemented with \ac{REST} and GraphQL interfaces?  
\end{description}