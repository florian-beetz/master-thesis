\section{Case Study}\label{sec:case-study}

\subsection{Case Study Design}

To evaluate the architectural differences between implementing a microservice architecture with REST and GraphQL interfaces, a case study was performed.
The case study implements an online shopping scenario, where users can place orders with multiple items~\raisebox{.5pt}{\textcircled{\raisebox{-.9pt} {1}}}.
When new orders are received, the inventory is checked~\raisebox{.5pt}{\textcircled{\raisebox{-.9pt} {2}}} if the items are available and then reserved for the user~\raisebox{.5pt}{\textcircled{\raisebox{-.9pt} {3}}}.
Additionally, the shipping cost is calculated~\raisebox{.5pt}{\textcircled{\raisebox{-.9pt} {4}}}.
When a payment is received from an external payment provider~\raisebox{.5pt}{\textcircled{\raisebox{-.9pt} {5}}}, the total of the order is checked~\raisebox{.5pt}{\textcircled{\raisebox{-.9pt} {6}}} and if the payment suffices, the order is marked ready for shipping~\raisebox{.5pt}{\textcircled{\raisebox{-.9pt} {7}}}~\raisebox{.5pt}{\textcircled{\raisebox{-.9pt} {8}}} and the reserved items can be booked out~\raisebox{.5pt}{\textcircled{\raisebox{-.9pt} {9}}}.
This process is shown in \autoref{fig:interaction}.

\begin{figure}[!htb]
    \centering
    \begin{tikzpicture}[
    node distance=3cm,
    comp/.style = {
        draw,
        circle,
        text width = 2cm,
        align = center,
        thick
    },
    l/.style = {
        text width = 3cm
    }
]

\node[comp, color=theme5, fill=theme5!10!white] (order) {Order Service};
\node[comp, below=of order, color=theme1, fill=theme1!10!white] (payment) {Payment Service};
\node[comp, above=of order, color=theme3, fill=theme3!10!white] (inventory) {Inventory Service};
\node[comp, right=of order, color=theme2, fill=theme2!10!white] (shipping) {Shipping Service};

\node[comp, left=of order] (client) {Client};
\node[comp, left=of payment] (paymentProcessor) {Payment Processor};

\draw[->, thick] (client) -- node[l, above, align=center] {\raisebox{.5pt}{\textcircled{\raisebox{-.9pt} {1}}} create order} (order);
\draw[->, thick] (order) to [bend left] node[l, left, align=right] {\raisebox{.5pt}{\textcircled{\raisebox{-.9pt} {2}}} check stock} (inventory);
\draw[->, thick] (order) to [bend right] node[l, right, align=left] {\raisebox{.5pt}{\textcircled{\raisebox{-.9pt} {3}}} reserve item} (inventory);
\draw[->, thick] (order) to [bend left] node[l, above, align=center] {\raisebox{.5pt}{\textcircled{\raisebox{-.9pt} {4}}} create shipment} (shipping);
\draw[->, thick] (order) to [bend right] node[l, left, align=right] {\raisebox{.5pt}{\textcircled{\raisebox{-.9pt} {5}}} create payment} (payment);
\draw[->, thick] (paymentProcessor) -- node[l, above, align=center] {\raisebox{.5pt}{\textcircled{\raisebox{-.9pt} {6}}} payment information} (payment);
\draw[->, thick] (payment) to [bend right] node[l, right, align=left] {\raisebox{.5pt}{\textcircled{\raisebox{-.9pt} {7}}} update order} (order);
\draw[->, thick] (order) to [bend right] node[l, below, align=center] {\raisebox{.5pt}{\textcircled{\raisebox{-.9pt} {8}}} ship order} (shipping);
\draw[->, thick] (shipping) to [bend right] node[l, right, align=left, text width=3.2cm] {\raisebox{.5pt}{\textcircled{\raisebox{-.9pt} {9}}} book out items} (inventory);

\end{tikzpicture}
    \caption{Service Interactions}
    \label{fig:interaction}
\end{figure}

\subsection{Implementation}

\subsection{Architectural Differences of the Implementations}

\subsection{Performance Comparison}

\subsection{Schema Evolution}

\subsection{Discussion}