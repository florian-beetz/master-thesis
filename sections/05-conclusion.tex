\section{Conclusion}\label{sec:conclusion}

This research aimed to identify architectural differences between microservices architectures using \ac{REST} and GraphQL for their interfaces.
Based on the performed case study, it can be concluded that for both approaches a similar architecture can be chosen.
Although both models follow different approaches to how they expose data to consumers, the available technologies, frameworks, and libraries mostly abstract this difference away.
Especially due to the comparable performance of both implementations, choosing one over the other is mostly a question of preference.

While this thesis is a first step in providing academic guidance on whether choosing to implement microservice architectures with \ac{REST} or GraphQL interface, the research certainly also has its limitations.
First of all, the case study was a purely academic project performed only as a part of this thesis, which threatens its applicability to real-world projects.
Based on this, the case study should be repeated on a larger real-world project.
Possible methods to further validate the findings of this thesis would be conducting expert interviews on the architectural differences of implemented \ac{REST} and GraphQL microservices.
Additionally, the chosen deployment strategy using containers probably introduced a performance overhead, threatening the validity of the performance evaluation.
While containers are a commonly used practice in real-world deployments, evaluating the performance on bare-metal hardware would strengthen the confidence in the collected data.
Lastly, the analyzed effort of evolving the schema largely relies on the nature of the introduced change.
In this case, only a single field was added to a resource.
It would be interesting to analyze the required effort of introducing such changes by deprecating or removing data fields.

In summary, this thesis contributed a starting point for analyzing the differences of microservice architectures implemented with \ac{REST} and GraphQL interfaces.
This thesis was not able to give a clear recommendation for choosing one technology over the other but concluded that both interface technologies can be implemented with a similar architecture for microservices.
Hence, it would probably be a good approach to provide both interfaces as it is done by big \ac{API} provides, such as GitHub\footnote{\url{https://github.blog/2016-09-14-the-github-graphql-api/}}.
This allows \ac{API} consumers to choose which technology they want to implement in clients while providers can reuse the same architecture.

%Future Work:
%* Repeat for a real-world case study: \cite{Runeson2012, Setzke2020, Goldkuhl2019, Myers1997}
%* Repeat more objectively: Implementation and Analysis were performed by the same person

