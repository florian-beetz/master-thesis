\section{Conceptual Foundations of Microservice Architectures}\label{sec:foundations}

\subsection{Microservices and Microservice Architectures}

Microservices and Microservice Architectures are a widely adopted architectural style originating from \acp{SOA}~\cite{Mazzara2020}.
In contrast to \ac{SOA}, microservices do not primarily focus on reuse and composition, but on independence, replacability, and autonomy~\cite{Baresi2020}.
Building an application using a microservice architecture describes the approach of decomposing the application into a set of small services (so-called \textit{microservices}), each running in their own process, providing a lightweight way of interacting with each other, often using \ac{HTTP} and \ac{REST}~\cite{Lewis2014}.
One service is responsible for a business capability.
Services can be deployed independantly and in an automated fashion.

Enterprise applications typically consist of a client-side \ac{UI}, a database, and a server-side application~\cite{Lewis2014}.
The server-side application handles user requests, executes domain logic, queries and updates the database, and populates views for the user.
Traditional applications following the monolithic architectural style package this functionality into a single logical executable, such that updates to any parts of the system require rebuilding and redeployment of the whole system.
Structuring this application only happens via language features such as classes, functions, and namespaces.
Horizontally scaling this application requires running multiple instances of the application behind a load-balancer.
While modularization can also be achieved with this monolithic style, microservices make the separation of modules explicit, by running in their own process.
They also allow scaling only the parts of the application requiring more resources.

According to Lewis and Fowler~\cite{Lewis2014}, there is no formal definition of microservice architecture, but common characteristics of the style can be identified.
However, not all microservice architectures will have all the characteristics, but most of them will exhibit most characteristics.

\paragraph{Componentization via Services}

Microservice architectures use services as their primary componentization mechanism~\cite{Lewis2014}.
Componentization via libraries and software modules can be called with in-memory function calls.
This contrasts out-of-process services, where communication is more expensive, as it requires using a mechanism such as web service requests, or \acp{RPC}.
However, this also makes services independantly deployable (see \autoref{fig:ms-componentization:mic}), whereas applications linked to a single executable must be built and deployed as a whole (see \autoref{fig:ms-componentization:mon}).
As a result, this allows better utilizing available resources as the parts of the application experiencing higher load can be horizontally scaled.
Additionally, changes to a service only require this service to be redeployed in most cases, while the rest of the application remains running.
Only changes to the interface of a service may require coordination between multiple services, however, this can also be avoided by cohesive service boundaries and evolution mechanisms in the service contracts.
However, having explicit service contracts requiring remote \ac{API} calls make it harder to breaking encapsulation than in-process function calls.

\begin{figure}[!htb]
    \centering
    \begin{subfigure}{.5\textwidth}
        \centering
        \begin{tikzpicture}[
    node distance=3cm,
    process/.style = {
        draw,
        rectangle
    },
    functionality/.style = {
        draw,
        circle,
        minimum width = 5mm
    },
    server/.style = {
        draw,
        rectangle
    },
    sline/.style = {
        line cap=round,
        line join=round
    }
]

\node[functionality, color=theme1, fill=theme1!10] (f1) {};
\node[functionality, right=3mm of f1, color=theme2, fill=theme2!10] (f2) {};
\node[functionality, below=3mm of f1, color=theme3, fill=theme3!10] (f3) {};
\node[functionality, right=3mm of f3, color=theme4, fill=theme4!10] (f4) {};
\node[process, fit=(f1)(f2)(f3)(f4)] (proc1) {};
\node[above=0 of proc1] (l1) {Process};
\node[server, fit=(proc1)(l1)] (srv1) {};
\draw[sline, fill=black!5] (srv1.north west) -- ++(0.25,0.25) -- ($(srv1.north east)+(0.25,0.25)$) -- (srv1.north east);
\draw[sline, fill=black!10] (srv1.north east) -- ++(0.25,0.25) -- ($(srv1.south east)+(0.25,0.25)$) -- (srv1.south east);
\node[below=0 of srv1] (l2) {Server A};


\node[functionality, right=2cm of f2, color=theme1, fill=theme1!10] (f5) {};
\node[functionality, right=3mm of f5, color=theme2, fill=theme2!10] (f6) {};
\node[functionality, below=3mm of f5, color=theme3, fill=theme3!10] (f7) {};
\node[functionality, right=3mm of f7, color=theme4, fill=theme4!10] (f8) {};
\node[process, fit=(f5)(f6)(f7)(f8)] (proc2) {};
\node[above=0 of proc2] (l3) {Process};
\node[server, fit=(proc2)(l3)] (srv2) {};
\draw[sline, fill=black!5] (srv2.north west) -- ++(0.25,0.25) -- ($(srv2.north east)+(0.25,0.25)$) -- (srv2.north east);
\draw[sline, fill=black!10] (srv2.north east) -- ++(0.25,0.25) -- ($(srv2.south east)+(0.25,0.25)$) -- (srv2.south east);
\node[below=0 of srv2] (l4) {Server B};

\end{tikzpicture}
        \caption{Monolith}
        \label{fig:ms-componentization:mon}
    \end{subfigure}%
    \begin{subfigure}{.5\textwidth}
        \centering
        \begin{tikzpicture}[
    node distance=3cm,
    process/.style = {
        draw,
        rectangle
    },
    functionality/.style = {
        draw,
        circle,
        minimum width = 5mm
    },
    server/.style = {
        draw,
        rectangle
    },
    sline/.style = {
        line cap=round,
        line join=round
    }
]

\node[functionality, color=theme1, fill=theme1!10] (f1) {};
\node[functionality, right=5mm of f1, color=theme2, fill=theme2!10] (f2) {};
\node[functionality, below=5mm of f1, color=theme2, fill=theme2!10] (f3) {};
\node[functionality, right=5mm of f3, color=theme4, fill=theme4!10] (f4) {};
\node[process, fit=(f1)] (proc1) {};
\node[process, fit=(f2)] (proc2) {};
\node[process, fit=(f3)] (proc3) {};
\node[process, fit=(f4)] (proc4) {};

\node[yshift=.5\baselineskip] (l1) at ($(proc1.north)!0.5!(proc2.north)$) {Processes};
\node[server, fit=(proc1)(proc2)(proc3)(proc4)(l1)] (srv1) {};
\draw[sline, fill=black!5] (srv1.north west) -- ++(0.25,0.25) -- ($(srv1.north east)+(0.25,0.25)$) -- (srv1.north east);
\draw[sline, fill=black!10] (srv1.north east) -- ++(0.25,0.25) -- ($(srv1.south east)+(0.25,0.25)$) -- (srv1.south east);
\node[below=0 of srv1] (l2) {Server A};



\node[functionality, right=2cm of f2, color=theme1, fill=theme1!10] (f5) {};
\node[functionality, right=5mm of f5, color=theme3, fill=theme3!10] (f6) {};
\node[functionality, below=5mm of f5, color=theme1, fill=theme1!10] (f7) {};
\node[functionality, right=5mm of f7, color=theme4, fill=theme4!10] (f8) {};
\node[process, fit=(f5)] (proc5) {};
\node[process, fit=(f6)] (proc6) {};
\node[process, fit=(f7)] (proc7) {};
\node[process, fit=(f8)] (proc8) {};

\node[yshift=.5\baselineskip] (l3) at ($(proc5.north)!0.5!(proc6.north)$) {Processes};
\node[server, fit=(proc5)(proc6)(proc7)(proc8)(l3)] (srv2) {};
\draw[sline, fill=black!5] (srv2.north west) -- ++(0.25,0.25) -- ($(srv2.north east)+(0.25,0.25)$) -- (srv2.north east);
\draw[sline, fill=black!10] (srv2.north east) -- ++(0.25,0.25) -- ($(srv2.south east)+(0.25,0.25)$) -- (srv2.south east);
\node[below=0 of srv2] (l4) {Server B};


\end{tikzpicture}
        \caption{Microservice}
        \label{fig:ms-componentization:mic}
    \end{subfigure}
    \caption{Componentization in monolithic and microservice architectures~\cite{Lewis2014}}
    \label{fig:ms-componentization}
\end{figure}

\paragraph{Organized around Business Capabilities}

Often, teams are organized after the architectural layer they are responsible for (\ac{UI} team, backend team, database team)~\cite{Kim2016, Lewis2014}.
Separation along these lines ignores the fact, that these architectural parts are usally tightly coupled and thus, cross-team projects are required to implement even simple changes.
Teams often try to implement logic into the part of the application they have access to, leading to business logic spread out into every part of the application.
This phenomenon is described by Conway's law:

\begin{displayquote}
    Any organization that designs a system (defined broadly) will produce a design whose structure is a copy of the organization's communication structure.
    \begin{flushright}
        --- \textit{Malvin E. Conway, 1968~\cite{Conway1968}}
    \end{flushright}
\end{displayquote}

When working with microservices, each service usually is developed by one team~\cite{Baresi2020, Mazzara2020, Lewis2014}.
Since services are organized around business capability, the team must have all required skills to develop the service, ranging from project management, over \ac{UI} development, to database development.
While this characteristic can also be applied to monolithic application development, the explicit separation required by microservices make it easier to form such teams.

\paragraph{Products not Projects}

Traditionally, development and operations teams of applications are clearly separated~\cite{Kim2016}.
Once the development of an application is done, it is handed over to an operations team and the developers start working on a different project.
Microservice proponents usually implement a DevOps model of developing and operating an application, where the same team owns a microservice over its whole lifecycle, maintaining and also operating it~\cite{Lewis2014}.
This moves away the focus of a project-based view, where software is seen as a set of functionalities, to a product-based view focusing on the business capabilities.
While this increases the contact developers have with their users, it also removes the organizational overhead of having separated development and operations teams.

\paragraph{Smart Endpoints and Dumb Pipes}

Microservices encapsulate all of the business logic inside of a service~\cite{Lewis2014}.
The communication mechanism is typically lightweight, such as \ac{REST}ful resource \acp{API} or lightweight message queues for asynchronous communication.
This heavily contrasts mechansims used in \ac{SOA}, such as \ac{ESB}, where apart from transportation, middlewares often facilitate routing, service choreography, message transformations, and applying business rules.
Lewis and Fowler~\cite{Lewis2014} name this approach \textit{smart endpoints and dumb pipes}.
It describes the idea, that a microservice should own its domain logic and be as decoupled and cohesive as possible.

\paragraph{Decentralized Governance}

Centralized Governance usually leads to standardization on one specific technology stack~\cite{Lewis2014}.
However, certain types of problems in an application might be more suited for solving in a different programming language or using a different type of database system.
Decentralizing the governance allows microservices architectures to follow a polyglot style and to pick the right tool for the job instead of working around the limitations of standards.
While in theory, every service in a microservices architecture can utilize a completely different technology stack and use a different language, in practice, there is usually a different kind of standards.
In a microservices architecture, some problems have to be solved for multiple services, such as data storage or inter-process communication.
This often leads to a kind of standardization by developing reusable libraries solving these problems.
Developing microservices based on shared libraries allows picking a different solution if apropriate, while also providing a basis to not start from scratch for every microservice.


\paragraph{Decentralized Data Management}

Decentralized data management essentially describes the idea of having different views on data in different contexts of a system~\cite{Lewis2014}.
Depending on the context, entities in a system may differ by their type, their attributes or their semantics.
In the context of microservices, this means that each microservice manages its own conceptual model of an entity.
Additionally, the decision of how data should be stored also lies at the service.
This allows each service to choose the most appropriate database technology.
One service for example might be implemented to use a traditional relational database, while another is implemented using a NoSQL database.

\begin{figure}[!htb]
    \centering
    \begin{subfigure}{.5\textwidth}
        \centering
        \begin{tikzpicture}[
    node distance=3cm,
    process/.style = {
        draw,
        rectangle
    },
    functionality/.style = {
        draw,
        circle,
        minimum width = 5mm
    },
    server/.style = {
        draw,
        rectangle
    },
    database/.style = {
        draw,
        cylinder,
        shape border rotate=90,
        aspect=0.25,
        text width = 1.5cm,
        align = center,
        cylinder uses custom fill, cylinder body fill=white, cylinder end fill=black!5
    },
    table/.style = {
        draw,
        rectangle,
        minimum width=5mm,
        minimum height=5mm
    }
]

\node[functionality, color=theme1, fill=theme1!10] (f1) {};
\node[functionality, right=3mm of f1, color=theme2, fill=theme2!10] (f2) {};
\node[functionality, below=3mm of f1, color=theme3, fill=theme3!10] (f3) {};
\node[process, fit=(f1)(f2)(f3)] (proc1) {};
\node[server, fit=(proc1)] (srv1) {};
\draw[fill=black!5] (srv1.north west) -- ++(0.25,0.25) -- ($(srv1.north east)+(0.25,0.25)$) -- (srv1.north east);
\draw[fill=black!10] (srv1.north east) -- ++(0.25,0.25) -- ($(srv1.south east)+(0.25,0.25)$) -- (srv1.south east);

\begin{pgfonlayer}{foreground}
    \node[table, below=2cm of f3, color=theme1, fill=theme1!10] (tab1) {};
    \draw[color=theme1] ($(tab1.north west)+(.05mm,0)$) -- ++(0,-1mm) -- ++(5mm,0) -- ++(0, -1mm) -- ++(-5mm,0)-- ++(0,-1mm) -- ++(5mm,0) -- ++(0, -1mm) -- ++(-5mm,0);
    \node[table, right=3mm of tab1, color=theme2, fill=theme2!10] (tab2) {};
    \draw[color=theme2] ($(tab2.north west)+(.05mm,0)$) -- ++(0,-1mm) -- ++(5mm,0) -- ++(0, -1mm) -- ++(-5mm,0)-- ++(0,-1mm) -- ++(5mm,0) -- ++(0, -1mm) -- ++(-5mm,0);
    \node[table, below=3mm of tab1, color=theme3, fill=theme3!10] (tab3) {};
    \draw[color=theme3] ($(tab3.north west)+(.05mm,0)$) -- ++(0,-1mm) -- ++(5mm,0) -- ++(0, -1mm) -- ++(-5mm,0)-- ++(0,-1mm) -- ++(5mm,0) -- ++(0, -1mm) -- ++(-5mm,0);
\end{pgfonlayer}
\node[database, fit=(tab1)(tab2)(tab3)] (db1) {};
\node[server, fit=(db1)] (srv1) {};
\draw[fill=black!5!white] (srv1.north west) -- ++(0.25,0.25) -- ($(srv1.north east)+(0.25,0.25)$) -- (srv1.north east);
\draw[fill=black!10!white] (srv1.north east) -- ++(0.25,0.25) -- ($(srv1.south east)+(0.25,0.25)$) -- (srv1.south east);

\draw[->] (db1) -- (proc1);

\end{tikzpicture}
        \caption{Monolith}
        \label{fig:sub1}
    \end{subfigure}%
    \begin{subfigure}{.5\textwidth}
        \centering
        \begin{tikzpicture}[
    node distance=3cm,
    process/.style = {
        draw,
        rectangle
    },
    functionality/.style = {
        draw,
        circle,
        minimum width = 5mm
    },
    server/.style = {
        draw,
        rectangle
    },
    sline/.style = {
        line cap=round,
        line join=round
    },
    database/.style = {
        draw,
        cylinder,
        shape border rotate=90,
        aspect=0.25,
        text width = 1.5cm,
        align = center,
        cylinder uses custom fill, cylinder body fill=white, cylinder end fill=black!5
    },
    table/.style = {
        draw,
        rectangle,
        minimum width=5mm,
        minimum height=5mm
    }
]

\node[functionality, color=theme1, fill=theme1!10] (f1) {};
\node[functionality, right=10mm of f1, color=theme2, fill=theme2!10] (f2) {};
\node[functionality, right=10mm of f2, color=theme3, fill=theme3!10] (f3) {};
\node[functionality, right=10mm of f3, color=theme3, fill=theme3!10] (f4) {};
\node[process, fit=(f1)] (proc1) {};
\node[process, fit=(f2)] (proc2) {};
\node[process, fit=(f3)] (proc3) {};
\node[process, fit=(f4)] (proc4) {};


\begin{pgfonlayer}{foreground}
    \node[table, below=8mm of f1, color=theme1, fill=theme1!10] (tab1) {};
    \draw[color=theme1] ($(tab1.north west)+(.05mm,0)$) -- ++(0,-1mm) -- ++(5mm,0) -- ++(0, -1mm) -- ++(-5mm,0)-- ++(0,-1mm) -- ++(5mm,0) -- ++(0, -1mm) -- ++(-5mm,0);
    \node[table, below=8mm of f2, color=theme2, fill=theme2!10] (tab2) {};
    \draw[color=theme2] ($(tab2.north west)+(.05mm,0)$) -- ++(0,-1mm) -- ++(5mm,0) -- ++(0, -1mm) -- ++(-5mm,0)-- ++(0,-1mm) -- ++(5mm,0) -- ++(0, -1mm) -- ++(-5mm,0);
    \node[table, color=theme3, fill=theme3!10] (tab3) at ($(f3.south)!0.5!(f4.south) - (0,1.75cm)$) {};
    \draw[color=theme3] ($(tab3.north west)+(.05mm,0)$) -- ++(0,-1mm) -- ++(5mm,0) -- ++(0, -1mm) -- ++(-5mm,0)-- ++(0,-1mm) -- ++(5mm,0) -- ++(0, -1mm) -- ++(-5mm,0);
\end{pgfonlayer}

\node[database, fit=(tab1)] (db1) {};
\node[database, fit=(tab2)] (db2) {};
\node[database, fit=(tab3)] (db3) {};

\node[server, fit=(proc1)(db1)] (srv1) {};
\draw[sline, fill=black!5!white] (srv1.north west) -- ++(0.25,0.25) -- ($(srv1.north east)+(0.25,0.25)$) -- (srv1.north east);
\draw[sline, fill=black!10!white] (srv1.north east) -- ++(0.25,0.25) -- ($(srv1.south east)+(0.25,0.25)$) -- (srv1.south east);

\node[server, fit=(proc2)(db2)] (srv2) {};
\draw[sline, fill=black!5!white] (srv2.north west) -- ++(0.25,0.25) -- ($(srv2.north east)+(0.25,0.25)$) -- (srv2.north east);
\draw[sline, fill=black!10!white] (srv2.north east) -- ++(0.25,0.25) -- ($(srv2.south east)+(0.25,0.25)$) -- (srv2.south east);

\node[server, fit=(proc3)] (srv3) {};
\draw[sline, fill=black!5!white] (srv3.north west) -- ++(0.25,0.25) -- ($(srv3.north east)+(0.25,0.25)$) -- (srv3.north east);
\draw[sline, fill=black!10!white] (srv3.north east) -- ++(0.25,0.25) -- ($(srv3.south east)+(0.25,0.25)$) -- (srv3.south east);

\node[server, fit=(proc4)] (srv4) {};
\draw[sline, fill=black!5!white] (srv4.north west) -- ++(0.25,0.25) -- ($(srv4.north east)+(0.25,0.25)$) -- (srv4.north east);
\draw[sline, fill=black!10!white] (srv4.north east) -- ++(0.25,0.25) -- ($(srv4.south east)+(0.25,0.25)$) -- (srv4.south east);

\node[server, fit=(db3)] (srv5) {};
\draw[sline, fill=black!5!white] (srv5.north west) -- ++(0.25,0.25) -- ($(srv5.north east)+(0.25,0.25)$) -- (srv5.north east);
\draw[sline, fill=black!10!white] (srv5.north east) -- ++(0.25,0.25) -- ($(srv5.south east)+(0.25,0.25)$) -- (srv5.south east);


\draw[->] (db1) -- (proc1);
\draw[->] (db2) -- (proc2);
\draw[->] (db3) -- (proc3);
\draw[->] (db3) -- (proc4);

\end{tikzpicture}
        \caption{Microservice}
        \label{fig:sub2}
    \end{subfigure}
    \caption{Data management in monolithic and microservice architectures~\cite{Lewis2014}}
    \label{fig:test}
\end{figure}

Traditional monolithic architectures tend to use just one single database and one database system.
This approach has the advantge over distributed data storage of microservices that transactions can be used to ensure consistency of data.
While implementing distributed transactions in microservices can be done, in practice an approach of eventual consistency is commonly chosen.
However, this requires additional effort to handle inconsistent or unavailable data at runtime and a way to fix errors that result from these consequences. 

\paragraph{Infrastructure Automation}

Infrastructure automation techniques provide automated builds, automatic tests, and automatic deployment to environments~\cite{Lewis2014}.
While all these techniques can also be applied to monolithic applications without any different challanges than microservices, a larger number of application components make it imperative to automate as much of the deployment process as possible.
This reduces the risk of errors caused by manually building or deploying services.

\paragraph{Design for Failure}

As a result of running a system consisting of multiple distributed processes, the system must be designed such that it can tolerate unavailability of services as gracefully as possible~\cite{Lewis2014}.
Calling services over the network introduces additional complexity that opens more possibilities for failures to be introduced in the system.
For example, the network could be unavailable or the host of a service could be unavailable.

In order to mitigate these failures a technique called chaos engineering has become popular~\cite{CEC2018}.
``Chaos Engineering is the discipline of experimenting on asystem in order to build confidence in the system's capability to withstand turbulent conditions in production''~\cite{CEC2018}.
Such experiments can for example consist of simulating server crashes, hard drive failures, or severed network connections.

Another important principle for microservices is extensive real-time monitoring~\cite{Lewis2014}.
Monitoring of microservices consists of collecting run-time metrics, such as average response time to an request, database requests per second, but also business metrics, for example orders per minute.

\paragraph{Evolutionary Design}

Whenever a system is split up into several components, the question arises how the system should be divided, as there are different possibilities regarding granularity of components and what belongs together~\cite{Lewis2014}.
The focus of componentization lies on independant replacability and upgradability, thus implying that a component should be rewritable without affecting its collaborators.
Proponents of microservice architectures often take this approach a step further and do not try to evolve a microservice but rather dispose of it when it needs rework or is no longer required~\cite{Fowler2016,Lewis2014}.
This also allows to write microservices that only meet one time-boxed business demand and are scrapped once the time frame has passed.

\subsection{Representational State Transfer (\acs{REST})}

\ac{REST} is an architectural style for network-based applications coined by Fielding~\cite{Fielding2000}.
Webservices and \acp{API} following this style are often referred to as \acs{REST}ful \acp{API}.
According to Fielding, this style focuses ``on the roles of components, the constraints upon their interaction with other components, and their interpretation of significant data elements''.
One of the key aspects of his definition is the word \textit{style}, because \ac{REST} is not a guideline or a standard~\cite{Malakhov2018}.

Typical \ac{REST} interactions consist of a user agent (e.g.\ a web browser) requesting a representation of a resource from a server~\cite{Erenkrantz2007}.
This communication be cached by intermediary proxies before being delivered.
The goal of this architecture is to reduce network latency, while also making the components more independent and scalable.

% confusion
In the past \ac{REST} has become an industry buzzword and is often confused with the notion of using \acs{HTTP}-based \acp{API} in general~\cite{Fielding2017}.
However, while \ac{HTTP} is commonly used together with \ac{REST} \acp{API}, the principles of \ac{REST} propose a set of more high-level constraints.

\subsubsection{REST Principles}

Although \ac{REST} does not provide a concrete set of rules or guidelines, several principles for \ac{REST}ful design exist.

\paragraph{Identifiable Resources}

The key abstraction of \ac{REST} are resources \cite{Fielding2000,Erenkrantz2007}.
Resources can be any information that can be named.
This includes types of information commonly found in the web, such as \acs{HTML}-documents, images, or JavaScript files, but also can mean non-virtual objects (e.g. a person), or temporal services (e.g. today's weather in Bamberg), collections of other resources, and so on.
While resources may change over time, the semantics of the resource must be static.
For example, in the context of a version control system, the resource \textit{latest version} always identifies the latest version, although this resource may change at some point from \textit{version 1.0} to \textit{version 1.1}.
Every resource is identified by a \textit{resource identifier}, e.g. a \ac{URL}. 

\paragraph{Self-descriptive Resources}

Components following the \ac{REST} architectural style communicate by using representations of resources \cite{Fielding2000}.
Representations of resources include some kind of binary data, and metadata describing this data.
In some cases, also metadata describing the metadata is part of the representation.
This separation of resource and representation introduces a layer of indirection between them \cite{Erenkrantz2007}.

\paragraph{Stateless Interaction}

According to Fielding, all \ac{REST} interactions are stateless \cite{Fielding2000}.
While this does not mean that \ac{REST} applications do not deal with state, it requires all requests to contain all the information for processing it to be contained in this request, independant of requests that may have preceeded it.
This allows applications to scale better, as no resources are required to maintain application state at the processing entity, while also allowing for requests to be processed in parallel.
Additionally, it allows for better caching of responses, as intermediaries can fully understand a single request in isolation.

\paragraph{Few Primitive Operations}

\ac{REST} components only support a very limited set of operations per resource \cite{Erenkrantz2007}.
The operations of a resource typically produce a representation of the resource capturing its current or intended state.
These operations are commonly implemented in practice using \ac{HTTP} methods~(see \autoref{sec:foundations:http}).

\paragraph{Cachability}

Caching is an important concept of \ac{REST}, as it improves the latency between client and server~\cite{Erenkrantz2007,Fielding2000}.
Caches can be used at the client-side to avoid sending network requests when the same resource is requested again, or at the server-side to avoid processing requests mutliple times.
This cachablity in \ac{REST} applications is achieved mainly by idempotency of certain requests.
Idempotency in this case means, that certain requests do not alter the future behaviour of the server.
So for example, if a client requests a resource multiple times, the response of the server does not change.

\paragraph{Transparent Intermediaries}

Between the components of \ac{REST}, intermediaries can be placed that are transparent to both the client and server \cite{Fielding2000}.
These intermediaries can take up different roles, for example cache communication between client and server, restrict access to resources, or modify and augment requests and responses.

\subsubsection{Hypertext Transfer Protocol (\acs{HTTP})}\label{sec:foundations:http}

Commonly, \ac{REST} \acp{API} are implemented on top of \ac{HTTP} for obtaining, creating, modifying, and deleting resources~\cite{Schermann2015}.
\ac{HTTP} is an application level protocol originally developed in 1990~\cite{RFC2068}.
Since 1997, it's version 1.1 was specified and since 2015, version 2 is available.
However, version 2 is only an alternative to 1.1 and does not make this version obsolete~\cite{RFC7540}.

\ac{HTTP} is a text-based protocol following the request-response scheme~\cite{RFC2068}.
All messages are separated into a header and body.
The header consists of a request or status line, depending on whether the message is a \ac{HTTP} request or response.
Additionally, the header may include one or more header fields.

Every request is made to a resource identified by an \ac{URL}~\cite{RFC2068}.
\ac{HTTP} specifies several so-called methods, that may be supported by a resource.
However, the set of methods is intentially left open-ended to allow extensions for different domains.
Resonses always include a status code in its status line indicating whether the request was successfull.
The status code is a three-digit number the server uses to communicate the state of the result of the request.
\ac{HTTP} defines the semantics of the status code to depend on the first digit of the number.
Codes starting with the digit \textit{1} indicate informational messages, \textit{2} indicates success, \textit{3} indicates redirections, \textit{4} indicates client errors, and \textit{5} indicates server errors.
However, while the specification includes a set of pre-defined status codes, this set is not conclusive and may be extended by implementations.

\autoref{lst:http-get-example} shows the general structure of a \ac{HTTP} request.
Line 1 is the request line indicating that this request is a \texttt{GET} request for the resource \texttt{/item/1} using the \ac{HTTP} version 1.1.
\texttt{GET} has the semantics of retrieving the information available, identified by the \ac{URL} of the request.
The header fields specify the server, the request is made to, and what kind of content the client expects.
The header then is terminated by an empty line and no request body is sent.

\begin{lstlisting}[caption={\acs{HTTP} GET request}, showlines=true, label=lst:http-get-example, language=http]
[*GET /item/1 HTTP/1.1*]
Host: localhost
Accept: application/json

\end{lstlisting}

A response to the above request could look as shown in \autoref{lst:http-response-example}.
The response indicates that the request was executed successfully with the status \texttt{200}.
The header fields indicate the server, that executed the request, and that it contains a body of 33 bytes that should be interpreted as \ac{JSON}.

\begin{lstlisting}[caption={\acs{HTTP} response to GET request}, label=lst:http-response-example, language=http]
[*HTTP/1.1 200 OK*]
Server: api-server
Content-Type: application/json
Content-Length: 33

{"title": "Item", "price": 1.99}
\end{lstlisting}

If the client now wants to update the resource requested in the above example, it would send a request as shown in \autoref{lst:http-put-example}.
This request uses the \texttt{PUT} method, which has the semantics of updating or creating the resource located at the given \ac{URL} based on the body of the request.
In this example, the request body contains the same \ac{JSON} representation of an item, but with an updated price.

\begin{lstlisting}[caption={\acs{HTTP} PUT request}, label=lst:http-put-example, language=http]
[*PUT /item/1 HTTP/1.1*]
Host: localhost
Content-Type: application/json
Content-Length: 33

{"title": "Item", "price": 3.59}
\end{lstlisting}

As already shown in the examples above, \ac{HTTP} defines a set of default methods that may be extended to suit the domain of implementations~\cite{RFC2068}.
\autoref{tab:http-methods} shows \ac{HTTP} methods most commonly implemented in the context of \ac{REST} \acp{API}~\cite{Buelthoff2019}.
\ac{HTTP} additionally specifies the properties \textit{safe} and \textit{idempotent} for methods.
If a method is safe, it should not have any side-effects for the user itself nor for other users.
Idempotent methods may have side-effects, but if a request uses an idempotent method and executed multiple times, the side-effects are the same as if it were just executed once.
Safe methods are important for caching, as intermediaries can infer, that the requested resource does not change as a result of the request.
Idempotency on the other hand is important for fault tolerance.
If a request times out, or the connectivity between client and server is lost, the request can be sent again, without causing unintended side-effects.

\begin{table}[ht]
    \centering
    \begin{tabular}{@{}rlcc@{}}
    \toprule
    \textbf{\acs{HTTP} Method}  & \textbf{Semantics}                    & \textbf{Idempotent}   & \textbf{Safe} \\ \midrule
    \texttt{GET}                & retrievs a resource                   & \checkmark            & \checkmark    \\
%    \texttt{HEAD}               & \checkmark            & \checkmark    \\
    \texttt{PUT}                & updates a resource                    & \checkmark            &               \\ 
    \texttt{DELETE}             & deletes a resource                    & \checkmark            &               \\ 
    \texttt{POST}               & \textit{dependant on implementation}  &                       &               \\
%    \texttt{PATCH}              &                       &               \\ 
%    \texttt{OPTIONS}            & \checkmark            & \checkmark    \\
    \bottomrule
    \end{tabular}
    \caption{Overview of selected HTTP Methods~\cite{RFC7321}}\label{tab:http-methods}
\end{table}

As shown in \autoref{tab:http-methods}, \texttt{POST} is a special method, as its semantics are mostly dependand on the implementation of the server~\cite{RFC2068}.
In practice, it is commonly used to implement \ac{RPC} style \acp{API} or to create new resources, for which their \acp{URL} are determined by the server after creation.

\subsubsection{\acf{JSON}}

While many \ac{REST} \acp{API} support multiple transport formats, 89\% of \acp{API} provide the choice to use \ac{JSON}, making it the most popular transport format~\cite{Buelthoff2019}.

\ac{JSON} is a text-based format, that was inspired by the JavaScript programming language~\cite{ECMAInternational2017}.
It provides a syntactic framework for data interchange, however, the semantics are entirely application specific.

Compared to \ac{XML}, \ac{JSON} has a very lightweight syntax, although its expressiveness is comparable.
\ac{JSON}'s main data structure is called an \textit{object}, which essentially is a nestable key-value data structure.
Such an object is shown in \autoref{lst:json-structure}.

\begin{lstlisting}[caption={\acs{JSON} data types and structures}, label=lst:json-structure, language=json]
{
    "boolean": true,
    "number": 42.0,
    "string": "Hello World",
    "array": [true, 1, "3rd element"],
    "nullValue": null
}
\end{lstlisting}

Apart from objects, \ac{JSON} provides an \textit{array} data type similar to many programming languages (see line 5 in \autoref{lst:json-structure}).
The primitive data types of \ac{JSON} are strings (line 4), numbers (line 3), and the boolean values \texttt{true} and \texttt{false} (line 2).
Additionally, \ac{JSON} provides the special \texttt{null} value, which is typically used to convey the semantics of no value.

In the context of \ac{REST} \acp{API}, several \ac{JSON} formats exist, that standardize the semantics of \ac{JSON} documents to a certain extend.
JSON-LD provides a way to encode linked data in standard \ac{JSON}~\cite{Kellogg2020}.
The \ac{HAL} provides a convention for defining hypermedia resources either as \ac{JSON} or \ac{XML} documents~\cite{HALdraft}.

\subsubsection{\acf{HATEOAS}}

\ac{HATEOAS} is another principle of \ac{REST}ful applications, although it is not very commonly implemented in \acp{API}~\cite{Liskin2011,Webber2010}.
A hypermedia system will exchange links as a part of the messages sent between the participants.
In an application following the \ac{HATEOAS} style, these links can be used to discover the application protocol, that is the valid transitions between different application states.
This discoverability is very common in human-computer interaction, but rarely found in completely computerized interactions.
A human might for example visit an online shop and browse the available items.
Once they have found an item, they place it in their shopping basket and head to check-out.
Notice, that in this interaction, the only resource identifier that needs to be known to the consumer is that of the initially visited homepage of the online shop.
All subsequent operations are executed by following links from the previously delivered website.
\ac{HATEOAS} describes exactly this kind of interaction.
During such an interaction, both the state of resources and the overall application state may change.
In the above example, the shopping basket resource of the user changes, as it contains an item as a result of the interaction, but also the application state changes, because after ordering the item, the item's stock has reduced by the ordered amount.

This kind of publishing the domain application protocol makes the available state transitions visible for the consumer, while keeping the business rules local to the owner of the resources~\cite{Webber2010}.
A consuming application does not need to infer any information about the state of a certain resource, and thus the possible operations on a resource, because the resource owner makes these operations visible.
Local assumptions about resource state might result in failing operations, because the client application shows invalid operations for the current state or hiding of valid operations.
Additionally, using links to advertise operations do not require the consumers of a service to make any assumptions about the structure of \acp{URL} of the service, thus allowing easier changes in their structure.

\subsection{GraphQL}

GraphQL is a query language for web \acp{API} originally developed in 2016 at Facebook~\cite{Hartig2017}.
The GraphQL framework consists of the query language itself, and a corresponding runtime executing queries~\cite{Wittern2019}.
Schemas describe the available operations and data types of an \ac{API}.
Clients then can exectly specify in a query what kind of data they want to obtain or mutate.

The specification of the required data by the client can lead to increased performance, as fewer request-response roundtrips are required and the response size is smaller compared to other \ac{API} paradigms~\cite{Wittern2019}.
Additionally, GraphQL does not require any additional description language to document interfaces and types of the \ac{API}, as the statically typed GraphQL schema is already available and can be queried with regular requests to the GraphQL interface through so-called introspection queries.

Recently, GraphQL is seeing a significant increase in usage~\cite{Hartig2017}.
Apart from big players like Coursera, GitHub, and Pinterest adopting GraphQL, Wittern also analyzed 8,399 GraphQL \acp{API} of open-source projects~\cite{Wittern2019}.
The following section describes the GraphQL framework in depth.

\subsubsection{Query Language}

GraphQL \acp{API} typically provide only one \ac{HTTP} endpoint, where requests can be made to~\cite{Schmidt2019,Facebook2018}.
However, the GraphQL specification intentionally does not require implementing server-client communication over a specific protocol.
This single endpoint accepts requests to retrieve, and change data, but also provides means to subscribe to certain events, that can be published asynchronously by the server.

This format of the requests and responses, however, is not directly encoded in the GraphQL query language, but an additional serialization format is required~\cite{Facebook2018}.
While the GraphQL specification states, that no specific serialization format is required, the most common one is \ac{JSON}.
An example of a GraphQL request and response serialized as \ac{JSON} is shown in \autoref{lst:graphql-serialized}.

\begin{listing}
    \centering
    \refstepcounter{listing}
    \noindent\begin{minipage}[b]{.4\textwidth}
    \begin{lstlisting}[language=json]
{
  "query": "query ...",
  "variables": { 
    "someId": 1 
  }
}
\end{lstlisting}
        \captionof{sublisting}{Serialized GraphQL request}
        \label{lst:graphql-request-serialized}            
        \end{minipage}%
    \begin{minipage}[b]{.6\textwidth}
        \begin{lstlisting}[language=json]
{ 
    "data": { ... },
    "errors": [ ... ],
    "extensions": { ... }
}
        \end{lstlisting}
        \captionof{sublisting}{Serialized response to a GraphQL query}
        \label{lst:graphql-response-serialized}            
    \end{minipage}
    \addtocounter{listing}{-1}
    \caption{\ac{JSON} as Serialization Format for GraphQL}
    \label{lst:graphql-serialized}
    \end{listing}

As shown in \autoref{lst:graphql-request-serialized}, the serialized request simply contains a \ac{JSON} object, where the field \texttt{query} contains the actual GraphQL query.
Additionally, the field \texttt{variables} can contain another \ac{JSON} object specifying variables that can be referenced in the query.
The response to a GraphQL query also is implemented as a \ac{JSON} object, where the actual response data is contained in the \texttt{data} field of the object.
Additionally, a response may contain a field \texttt{errors}, where details about potential errors can be included.
The field \texttt{extensions} is intended for implementation specific details.

\begin{listing}
    \centering
    \refstepcounter{listing}
    \noindent\begin{minipage}[b]{.4\textwidth}
    \begin{lstlisting}[language=graphql]
query {
  person(personID: 1) {
    name,
    filmConnection {
      films {
        title
      }
}}}
\end{lstlisting}
        \captionof{sublisting}{GraphQL request}
        \label{lst:graphql-request}            
        \end{minipage}%
    \begin{minipage}[b]{.6\textwidth}
        \begin{lstlisting}[language=json]
{ "data": { "person": {
  "name": "Luke Skywalker",
  "filmConnection": { "films": [
    {"title":" A New Hope"},
    {"title":" The Empire Strikes Back"},
    {"title":" Return of the Jedi"},
    {"title":" Revenge of the Sith"}
]}}}}
        \end{lstlisting}
        \captionof{sublisting}{Response to the request}
        \label{lst:graphql-response}            
    \end{minipage}
    \addtocounter{listing}{-1}
    \caption[GraphQL request to SWAPI and its response]{GraphQL request to SWAPI\footnotemark and its response}
    \label{lst:graphql}
    \end{listing}
    
\footnotetext{The StarWars API: \url{https://swapi.dev/}}

\begin{itemize}
    \item two different languages: Query and Schema definition
    \item both roughly based on JSON, but typed!
    \item explain structure of queries
\end{itemize}

\subsubsection{GraphQL Schemas}

\begin{lstlisting}[caption={GraphQL type definition}, language=graphqls]
type Character {
    name: String!
    appearsIn: [Episode!]!
}
\end{lstlisting}

\begin{lstlisting}[caption={Arguments on GraphQL fields}, language=graphqls]
type Starship {
    id: ID!
    name: String!
    length(unit: LengthUnit = METER): Float
}
\end{lstlisting}

\begin{lstlisting}[caption={Query and Mutation types}, language=graphqls]
type Query {
    hero(episode: Episode): ~Character
    droid(id: ID!): Droid
}

type Mutation { 
    # [...]
}

schema {
    ~query: ~Query
    ~mutation: ~Mutation
}      
\end{lstlisting}

\begin{itemize}
    \item what constructs are available for Schemas?
    \item extensible through custom scalars!
    \item directives
\end{itemize}

\subsubsection{Tools for GraphQL}

\begin{itemize}
    \item GrapiQL
    \item GraphQL Playground
    \item Code Generators
\end{itemize}