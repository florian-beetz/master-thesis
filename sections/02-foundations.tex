\section{Conceptual Foundations of Microservices Architectures}\label{sec:foundations}

\subsection{Microservices and Microservices Architectures}

\subsection{Representational State Transfer (\acs{REST})}

\ac{REST} is an architectural style for network-based applications coined by Fielding~\cite{Fielding2000}.
Webservices and \acp{API} following this style are often referred to as \acs{REST}ful \acp{API}.
According to Fielding, this style focuses ``on the roles of components, the constraints upon their interaction with other components, and their interpretation of significant data elements''.
One of the key aspects of his definition is the word \textit{style}, because \ac{REST} is not a guideline or a standard~\cite{Malakhov2018}.

Typical \ac{REST} interactions consist of a user agent (e.g.\ a web browser) requesting a representation of a resource from a server~\cite{Erenkrantz2007}.
This communication be cached by intermediary proxies before being delivered.
The goal of this architecture is to reduce network latency, while also making the components more independent and scalable.

% confusion
In the past \ac{REST} has become an industry buzzword and is often confused with the notion of using \acs{HTTP}-based \acp{API} in general~\cite{Fielding2017}.
However, while \ac{HTTP} is commonly used together with \ac{REST} \acp{API}, the principles of \ac{REST} propose a set of more high-level constraints.

\subsubsection{REST Principles}



\subsubsection{Hypertext Transfer Protocol (\acs{HTTP})}

\begin{table}[ht]
    \centering
    \begin{tabular}{@{}rcc@{}}
    \toprule
    \textbf{\acs{HTTP} Method}  & \textbf{Idempotent}   & \textbf{Safe} \\ \midrule
    \texttt{GET}                & \checkmark            & \checkmark    \\
    \texttt{HEAD}               & \checkmark            & \checkmark    \\
    \texttt{POST}               &                       &               \\
    \texttt{PUT}                & \checkmark            &               \\ 
    \texttt{DELETE}             & \checkmark            &               \\ 
    \texttt{PATCH}              &                       &               \\ 
    \texttt{OPTIONS}            & \checkmark            & \checkmark    \\ \bottomrule
    \end{tabular}
    \caption{Overview of selected HTTP Methods~\cite{RFC7321}}\label{tab:http-methods}
\end{table}

\subsubsection{Javascript Object Notation (\acs{JSON})}

\subsubsection{Hypertext as the Engine of Application State (\acs{HATEOAS})}

\subsection{GraphQL}

\subsubsection{Structure of the Language}

\subsubsection{GraphQL Schemas}

\subsubsection{Tools for GraphQL}