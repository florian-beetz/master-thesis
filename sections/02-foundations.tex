\section{Conceptual Foundations of Microservices Architectures}\label{sec:foundations}

\subsection{Microservices and Microservices Architectures}

\subsection{Representational State Transfer (\acs{REST})}

\ac{REST} refers to an architectural style coined by Fielding \cite{Fielding2000}.
% todo: wide adoption

Webservices and \acp{API} following this style are often referred to as \acs{REST}ful \acp{API}.


\subsubsection{REST Principles}

\subsubsection{Hypertext Transfer Protocol (\acs{HTTP})}

\begin{table}[ht]
    \centering
    \begin{tabular}{@{}lll@{}}
    \toprule
    \textbf{\acs{HTTP} Method}  & \textbf{Idempotent}   & \textbf{Safe} \\ \midrule
    \texttt{GET}                & \checkmark            & \checkmark    \\
    \texttt{HEAD}               & \checkmark            & \checkmark    \\
    \texttt{POST}               &                       &               \\
    \texttt{PUT}                & \checkmark            &               \\ 
    \texttt{DELETE}             & \checkmark            &               \\ 
    \texttt{PATCH}              &                       &               \\ 
    \texttt{OPTIONS}            & \checkmark            & \checkmark    \\ \bottomrule
    \end{tabular}
    \caption{Overview of selected HTTP Methods \cite{RFC7321}}\label{tab:http-methods}
\end{table}

\subsubsection{Javascript Object Notation (\acs{JSON})}

\subsubsection{Hypertext as the Engine of Application State (\acs{HATEOAS})}

\subsection{GraphQL}

\subsubsection{Structure of the Language}

\subsubsection{GraphQL Schemas}

\subsubsection{Tools for GraphQL}