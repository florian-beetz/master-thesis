\section{Conceptual Foundations of Microservices Architectures}\label{sec:foundations}

\subsection{Microservices and Microservices Architectures}

\subsection{Representational State Transfer (\acs{REST})}

\ac{REST} is an architectural style for network-based applications coined by Fielding~\cite{Fielding2000}.
Webservices and \acp{API} following this style are often referred to as \acs{REST}ful \acp{API}.
According to Fielding, this style focuses ``on the roles of components, the constraints upon their interaction with other components, and their interpretation of significant data elements''.
One of the key aspects of his definition is the word \textit{style}, because \ac{REST} is not a guideline or a standard~\cite{Malakhov2018}.

% confusion


\subsubsection{REST Principles}

Although \ac{REST} does not provide a concrete set of rules or guidelines, several principles for \ac{REST}ful design exist.

\paragraph{Identifiable Resources}

The key abstraction of \ac{REST} are resources \cite{Fielding2000,Erenkrantz2007}.
Resources can be any information that can be named.
This includes types of information commonly found in the web, such as \acs{HTML}-documents, images, or JavaScript files, but also can mean non-virtual objects (e.g. a person), or temporal services (e.g. today's weather in Bamberg), collections of other resources, and so on.
While resources may change over time, the semantics of the resource must be static.
For example, in the context of a version control system, the resource \textit{latest version} always identifies the latest version, although this resource may change at some point from \textit{version 1.0} to \textit{version 1.1}.
Every resource is identified by a \textit{resource identifier}, e.g. a \ac{URL}. 

\paragraph{Self-descriptive Resources}

Components following the \ac{REST} architectural style communicate by using representations of resources \cite{Fielding2000}.
Representations of resources include some kind of binary data, and metadata describing this data.
In some cases, also metadata describing the metadata is part of the representation.
This separation of resource and representation introduces a layer of indirection between them \cite{Erenkrantz2007}.

\paragraph{Stateless Interaction}

\paragraph{Few Primitive Operations}

\paragraph{Cachability}

\paragraph{Intermediaries}

\subsubsection{Hypertext Transfer Protocol (\acs{HTTP})}

\begin{table}[ht]
    \centering
    \begin{tabular}{@{}rcc@{}}
    \toprule
    \textbf{\acs{HTTP} Method}  & \textbf{Idempotent}   & \textbf{Safe} \\ \midrule
    \texttt{GET}                & \checkmark            & \checkmark    \\
    \texttt{HEAD}               & \checkmark            & \checkmark    \\
    \texttt{POST}               &                       &               \\
    \texttt{PUT}                & \checkmark            &               \\ 
    \texttt{DELETE}             & \checkmark            &               \\ 
    \texttt{PATCH}              &                       &               \\ 
    \texttt{OPTIONS}            & \checkmark            & \checkmark    \\ \bottomrule
    \end{tabular}
    \caption{Overview of selected HTTP Methods~\cite{RFC7321}}\label{tab:http-methods}
\end{table}

\subsubsection{Javascript Object Notation (\acs{JSON})}

\subsubsection{Hypertext as the Engine of Application State (\acs{HATEOAS})}

\subsection{GraphQL}

\subsubsection{Structure of the Language}

\subsubsection{GraphQL Schemas}

\subsubsection{Tools for GraphQL}