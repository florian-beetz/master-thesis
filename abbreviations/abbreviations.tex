%
% Abk�rzungen
%
\section*{Abbreviations}
% In Klammern steht das l�ngste Akronym!
\begin{acronym}[HATEOAS]
    \acro{API}{Application Programming Interface}
    \acro{CSRF}{Cross-Site Request Forgery}
    \acro{DNS}{Domain Name Service}
    \acro{ESB}{Enterprise Service Bus}
    \acro{HAL}{Hypertext Application Language}
    \acro{HATEOAS}{Hypermedia as the Engine of Application State}
    \acro{HTML}{Hypertext Markup Language}
    \acro{HTTP}{Hypertext Transfer Protocol}
    \acro{IaC}{Infrastructure as Code}
    \acro{IDE}{Integrated Development Environment}
    \acro{JDK}{Java Development Kit}
    \acro{IdP}{Identity Provider}
    \acro{JPA}{Java Persistence \acs{API}}
    \acro{JSON}{Javascript Object Notation}
    \acro{JVM}{Java Virtual Machine}
    \acro{JWT}{\acs{JSON} Web Token}
    \acro{MVC}{Model View Controller}
    \acro{OAS}{Open\acs{API} Specification}
    \acro{OIDC}{OpenID Connect}
    \acro{REST}{Representational State Transfer}
    \acro{RP}{Relying Party}
    \acro{RPC}{Remote Procedure Call}
    \acro{RQ}{Research Question}
    \acro{SAML}{Security Assertion Markup Language}
    \acro{SOA}{Service-Oriented Architecture}
    \acro{SSO}{Single Sign-On}
    \acro{SQL}{Structured Query Language}
    \acro{SWAPI}{Star Wars \acs{API}}
    \acro{TLS}{Transport Layer Security}
    \acro{UI}{User Interface}
    \acro{URI}{Uniform Resource Identifier}
    \acro{URL}{Uniform Resource Locator}
    \acro{XML}{Extensible Markup Language}
\end{acronym}