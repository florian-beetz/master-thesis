%===============================================================================
% Zentrale Layout-Angaben und Befehle
%===============================================================================
%
% F???r bessere Sicht von falschen Umbr???chen die Option draft benutzen.
% Dadurch k???nnen aber die eingebundenen Bilder nicht sichtbar sein.
\documentclass[a4paper, 12pt]{article}
%
% Hier zun???chst die ben???tigten Packages
%\usepackage[utf-8]{inputenc}
\usepackage{fancyhdr}
\usepackage[T1]{fontenc}
\usepackage{ae}
\usepackage{listings}
\usepackage{color}
\usepackage{xcolor}
\usepackage{listings}
\usepackage{newfloat,caption}
\usepackage{subcaption}
\usepackage{wrapfig}
\usepackage[printonlyused]{acronym}
\usepackage[hyphens]{url}
\usepackage{hyperref}
%
% Einbindung des Grafik-Pakets
%\ifx\pdfoutput\undefined
%	\usepackage[dvips]{graphicx}
%\else
%	\usepackage[pdftex]{graphicx}
%\pdfcompresslevel=9
%\pdfpageheight=297mm
%\pdfpagewidth=210mm
%\fi
\usepackage{graphicx}
\usepackage{enumitem}
%
% Page-Layout
\setlength\headheight{14pt}
\setlength\topmargin{-15,4mm}
\setlength\oddsidemargin{-0,4mm}
\setlength\evensidemargin{-0,4mm}
\setlength\textwidth{160mm}
\setlength\textheight{252mm}
%
% Absatzeinstellungen
\setlength\parindent{0mm}
\setlength\parskip{2ex}
%
% Kopf- und Fusszeile
\pagestyle{fancy}
\fancyhf{} % alles l�schen
\fancyhead[LO]{\footnotesize\sc\nouppercase{\leftmark}}
\fancyfoot[LO]{\footnotesize\sc Lehrstuhl f�r Praktische Informatik}
\fancyfoot[RO]{\thepage}
\renewcommand{\headrulewidth}{0pt}
\renewcommand{\footrulewidth}{0pt}
%
% Bessere Fehlermeldungen
\errorcontextlines=999
%
% Anweisung zur Erstellung der Titelseite
% #1 Bachelorarbeit || Masterarbeit
% #2 = Studiengang
% #3 = Titel der Arbeit
% #4 = Autor
% #5 = Abgabedatum
\renewcommand{\maketitle}[5]
{
\pagenumbering{Alph}
\begin{titlepage}
\centering
\begin{minipage}[t]{16cm}
\begin{minipage}{3cm}
    \includegraphics[height=26mm]{includes/vs-logo}
\end{minipage}
\hfill
\begin{minipage}{9cm}
  \centering
    Otto-Friedrich-Universit�t Bamberg\\[12pt]
    {\Large Lehrstuhl f�r Praktische Informatik}
\end{minipage}
\hfill
\begin{minipage}{3cm}
    \includegraphics[height=26mm]{includes/UB-Logo-neu_blau-cmyk}
\end{minipage}
\end{minipage}\\[130pt]
{\LARGE #1}\\[24pt]
im Studiengang #2\\
der Fakult�t Wirtschaftsinformatik und Angewandte Informatik\\
der Otto-Friedrich-Universit�t Bamberg\\[90pt]
Zum Thema:\\[24pt]
{\Huge #3}\\[60pt]
\vfill
\begin{minipage}{\textwidth}
\center
Vorgelegt von:\\
{\Large #4\\[12pt]}
Themensteller:\\
Prof. Dr. Guido Wirtz\\[12pt]
Abgabedatum:\\
#5\\
\end{minipage}
\end{titlepage}
}
%
% Anweisung zur Erstellung der Eigenst�ndigkeitserkl�rung
% #1 = Typ der Arbeit
% #2 = Datum
% #3 = Vorname Name
\newcommand{\makedeclaration}[3]
{
	\fancyhead[LO]{\footnotesize\sc\nouppercase{Eigenst�ndigkeitserkl�rung}}
	%
			\vspace*{18cm}
			Ich erkl�re hiermit gem�� �17 Abs. 2 APO, dass ich die vorstehende #1 selbstst�ndig verfasst 
			und keine anderen als die angegebenen Quellen und Hilfsmittel verwendet habe.\\
			\vspace*{1cm}
			
			Bamberg, den #2 \hspace{5cm} #3
	%
}
%
% Wird f�r Hintergrund von Codelistings ben�tigt
\definecolor{hellgrau}{gray}{0.9}
%
% Einstellungen f�r Java-Code
\lstdefinestyle{javaStyle}{%
  basicstyle=\small,%
  backgroundcolor=\color{hellgrau},%
  keywordstyle=\bfseries,%
  showstringspaces=false,%
  language=Java,%
  numbers=left,%
  numberstyle=\tiny,%
  stepnumber=1,%
  numbersep=5pt,%
  extendedchars=true,%
  xleftmargin=2em,%
  lineskip=-1pt,%
  breaklines%
}
%
% neues environment f�r Java-Sourcecode
% #1 = "caption={Hier eigene �berschrift}, label={Hier eigenes Label}"
\lstnewenvironment{javacode}[1][]{%
\lstset{style=javaStyle,#1}%
}{}
%
% Befehl zum Einbinden von Java-Sourcecode aus Datei
% #1 = Dateiname relativ zu src-Verzeichnis
% #2 = �berschrift
% #3 = Label
\newcommand{\javafile}[3]{%
   \lstinputlisting[%
     caption={#2},%
     label={#3},%
     style=javaStyle]{src/#1}%
}
%
% Einbindung eines Bildes
% #1 = label f�r \ref-Verweise
% #2 = Name des Bildes ohne Endung relativ zu images-Verzeichnis
% #3 = Beschriftung
% #4 = Breite des Bildes im Dokument in cm
\newcommand{\bild}[4]{%
  \begin{figure}[htb]%
    \begin{center}%
      \includegraphics[width=#4cm]{images/#2}%
      \vskip -0.3cm%
      \caption{#3}%
      \vskip -0,2cm%
      \label{#1}%
    \end{center}%
  \end{figure}%
}
%
% Umgebung f�r Fliesstext um Grafik
% #1 = Ausrichtung: r, l, i, ...
% #2 = Breite des Bildes in cm
% #3 = Name des Bildes ohne Endung relativ zu images-Verzeichnis
% #4 = Beschriftung
% #5 = label f�r \ref-Verweise
\newcommand{\fliesstext}[5]{%
\begin{wrapfigure}{#1}{#2cm}%
\includegraphics[width=#2cm]{images/#3}%
\caption{#4}%
\label{#5}%
\end{wrapfigure}%
}
%%% Local Variables:
%%% mode: latex
%%% TeX-master: t
%%% End:
